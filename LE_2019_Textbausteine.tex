% Copyright 2019
% IfV NRW, Fachhochschule Südwestfalen
% Arbeitsgebiet Mediengestaltung und Publishing
%
% Diese Datei wird eingesetzt für die Erstellung von Lerneinheiten im Verbundstudium.

% 2019-10-15
% Sandra Ciupka


%%%%%%%%%% Abbildungen %%%%%%%%%%%%%%%%%%%%%%%%%%%%%%%%%%%%%%%%%%%%%%%%%%%%%%%%%%%%%%%%%%%%%%%%%%%%%%%%%%%%
% Abbildung: 107 mm
\begin{figure}[!h]
	\includegraphics[width=10.7cm]{Dateipfad}
	\caption{Beschriftung}
	\label{Bezeichnung für Verweis}
\end{figure}


% Abbildung: 163 mm
\begin{figure}[!h]
	\begin{wide}	
		\includegraphics[width=\textwidth]{Dateipfad}
		\begin{minipage}{0.66\textwidth}
			\caption{Beschriftung}
			\label{Bezeichnung}
		\end{minipage}
	\end{wide}
\end{figure}

Referenz auf Abbildung:
Abbildung~\ref{Bezeichnung für Verweis} [S.\pageref{Bezeichnung für Verweis}]


%%%%%%%%%% Tabellen %%%%%%%%%%%%%%%%%%%%%%%%%%%%%%%%%%%%%%%%%%%%%%%%%%%%%%%%%%%%%%%%%%%%%%%%%%%%%%%%%%%%%%%
% Tabelle mit gleichbreiten Spalten
\begin{table}
	\captionabove{Tabellenbeschriftung}
	\begin{tabularx}{\textwidth}{|X|X|}
		\hline
		\bfseries Zelle links & \bfseries Zelle rechts \\ % Tabellenkopf fett
		\hline
		Zelle 1 & Zelle 2\\
		\hline
		Zelle 2 & Zelle 3\\
		\hline	
	\end{tabularx}
\end{table}

% Tabelle mit flexiblen Spalten
\begin{table}
	\captionabove{Tabellenbeschriftung}
	\begin{tabulary}{10.7cm}{|L|L|L|L|L|L|} % linksbündig
		\hline
		Zelle 1 & 2	& 3 & Zelle 4 & 5 & 6\\
		\hline
		Zelle 7	& 8	& 9 & Zelle 10 & 11 & 12\\
		\hline
	\end{tabulary}
\end{table}

% Breite Tabelle (163mm)
\begin{table}
	\captionabove{Tabellenbeschriftung}
	\begin{tabulary}{16.3cm}{|L|L|L|}
		\hline
		Dies ist Platzhaltertext für Zelle 1 & Dies ist Platzhaltertext für Zelle 2 & Dies ist Platzhaltertext für Zelle 3\\
		\hline
		Dies ist Platzhaltertext für Zelle 4 & Dies ist Platzhaltertext für Zelle 5 & Dies ist Platzhaltertext für Zelle 6\\
		\hline
	\end{tabulary}
\end{table}


%%%%%%%%%% Textgruppen %%%%%%%%%%%%%%%%%%%%%%%%%%%%%%%%%%%%%%%%%%%%%%%%%%%%%%%%%%%%%%%%%%%%%%%%%%%%%%%%%%%%

%%%%%%%%%% Marginalie/Randbemerkung
\mpar{Text}


%%%%%%%%%% Fußnote
\footnote{Text}


%%%%%%%%%% Rasterfläche
\begin{MyToDo} % optional [Überschrift]
	Dies ist Platzhaltertext.
\end{MyToDo}\\


%%%%%%%%%% Text eingezogen, kursive Schrift
\begin{Einzugit}{Definition}
	Dies ist Platzhaltertext.	
\end{Einzugit}


%%%%%%%%%% Text eingezogen, normale Schrift
\begin{Einzugup}{Satz, Beispiel}
	Dies ist Platzhaltertext.	
\end{Einzugup}


%%%%%%%%%% Textgruppe mit Überschrift
\begin{Textgruppe}{Überschrift Textgruppe}
	Dies ist Platzhaltertext.
\end{Textgruppe}